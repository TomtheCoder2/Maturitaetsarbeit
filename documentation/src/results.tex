Describing the results is not an easy task, as the project is still in development and the final results are not yet available.
However, some general results can be presented, such as the construction of the models, the embedded programming, and the image processing and optics.

\subsection{Embedded Programming}\label{subsec:results_embedded}
The code on the Arduino is not as sophisticated as desired.
While it performs the tasks well enough at low speeds, at higher speeds the motor becomes inconsistent and inaccurate.
This issue can be resolved with the custom PCB provided by the company, which uses the RS485 communication standard.
Therefore, this improvement is listed as a future enhancement in Section~\ref{sec:improvements}.

\subsubsection{Precision}\label{subsubsec:precision}
The precision has not yet reached $1\,\mathrm{mm}$.
Measuring it precisely is challenging because the closer the player is, the more precise the IR sensor needs to be.
Additionally, the IR sensor occasionally fails to measure the distance correctly, leading to improper movement of the player.

\subsection{Image Processing and Optics}\label{subsec:results_image}
The undistortion function works very well, largely due to over 20 iterations of the calibration process.
Ball detection is also highly accurate, successfully detecting only the ball while avoiding the tube or the player.
However, under certain lighting conditions, the player’s hand is occasionally detected as the ball.
This issue could be resolved by using a differently colored ball.

The ball prediction is accurate, with a small margin of error.
However, timing inconsistencies remain, where the player sometimes shoots too early or too late.
This issue is attributed to camera latency and USB driver delays.
Additionally, the inexpensive physical DC motor driver used is outdated, unable to ramp up speed quickly enough, and prone to overheating.

\subsection{General}\label{subsec:general}
Overall, this project successfully demonstrates a working proof of concept for a foosball robot.
Its main advantage over similar projects lies in the simplicity of its design, which eliminates the need for motor movement.
This simplicity allows for easy expansion to heavier, faster, and stronger motors.

