\section{Abstract}\label{sec:abstract}
\todo{Rewrite this part/adapt it to the achieved results}
The question of whether a machine can beat a human in a specific task is as old as the first machines.
Machines have already surpassed humans in many areas, such as calculators, chess, and even driving cars (even though the laws do not allow it yet).
But what about foosball?
This project aims to build a machine that can beat a human in foosball.
The machine will use a camera to film the game from below and will have two motors for each axis to move the players and shoot the ball.
A computer will control the machine, using a neural network to determine the best move.
This is quite ironic, as the human brain itself is a neural network with over 80 billion neurons.
The success of this project could provide new insights into the capabilities of AI in fast-paced, real-time environments.


\section{Introduction}\label{sec:introduction}

Whether a machine can surpass a human in a specific task has fascinated people since the first machines were created.
Over time, machines have outperformed humans in many areas, from simple calculations to complex strategy games like chess.
With advancements in AI, machines are now capable of tasks like driving cars, though they’re not widely used yet due to legal restrictions.

But can a machine beat a human in a game of foosball?
Unlike chess, foosball is fast-paced and unpredictable, requiring quick reflexes and real-time decision-making.
This makes it a perfect challenge for testing the limits of AI and robotics.

The goal of this project is to build a foosball-playing machine that can outperform a human.
The machine will rely on a camera to monitor the game from below and will be controlled by a computer using two motors per axis to move the players and shoot the ball.
The computer will use a neural network to determine the best moves—an interesting parallel, since the human brain is essentially a large neural network with billions of neurons.
This project will explore how AI can handle complex, dynamic tasks like foosball and what this means for the future of AI .


\section{Motivation}\label{sec:motivation}
I love the combination of hardware and software—building a machine and then writing intelligent software to control it.
I also love playing foosball, so I thought it would be a great idea to combine these two interests.
The hardware itself will be challenging, but the software will be even more demanding, as it requires inventing new strategies that demand inhuman precision and speed.
The dream of playing foosball with precise, controlled moves would unlock a whole new level of strategy and fun.
Additionally, the machine could be used to teach and train people to play foosball, providing an innovative way to learn the game.
It would also enable a solo mode where you can play against the machine at different difficulty levels, which would be an excellent way to practice and improve your skills.
