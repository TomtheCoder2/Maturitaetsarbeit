\section*{~}
Whether a machine can surpass a human in a specific task has fascinated people since the first machines were created.
Over time, machines have outperformed humans in many areas, from simple calculations to complex strategy games like chess.
With advancements in AI, machines are now capable of tasks like driving cars.
But can a machine beat a human in a game of foosball (table soccer)?
Unlike chess, foosball is fast-paced and unpredictable, requiring quick reflexes and real-time decision-making.
This makes it a perfect challenge for testing the limits of AI and robotics.
The goal of this project is to build a foosball-playing machine that can play with a human.
The machine will rely on a camera to monitor the game from below and will be controlled by a computer using two motors per axis to move the players and shoot the ball.
The computer will use a neural network to determine the best moves—an interesting parallel, since the human brain is essentially a large neural network with billions of neurons.
The goal of this project is to build and test a foosball goalkeeper robot as a prove of concept, that a machine could compete with a human player.

\section{Motivation}\label{sec:motivation}
I love the combination of hardware and software—building a machine and then writing intelligent software to control it.
I also love playing foosball, so I thought it would be a great idea to combine these two interests.
The hardware itself will be challenging, but the software will be even more demanding, as it requires inventing new strategies that demand inhuman precision and speed.
The dream of playing foosball with precise, controlled moves would unlock a whole new level of strategy and fun.
Additionally, the machine could be used to teach and train people to play foosball, providing an innovative way to learn the game.
It would also enable a solo mode where a player could play against the machine at different difficulty levels, which would be an excellent way to practice and improve skills.
