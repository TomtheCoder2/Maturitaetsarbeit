% Preamble
\documentclass[11pt,openany, english]{book}
\usepackage[a4paper, total={6in, 8in},headsep=0.5cm]{geometry}

% Packages
\usepackage{amsmath}
\usepackage{tikz}
\usetikzlibrary{shapes.geometric, arrows}
\usetikzlibrary{decorations.markings}
\usetikzlibrary{decorations.pathmorphing,patterns}
\usepackage{wrapfig}
%\usepackage{amstex}
\usetikzlibrary{calc}
\usepackage{fancyheadings}
\usepackage{mathtools}
\usepackage{transparent}
\usepackage{graphicx}
\usepackage{float}
\usepackage{subcaption}
\usepackage [autostyle, english = american]{csquotes}
\usepackage{hyperref}
\hypersetup{
    colorlinks,
    linkcolor={blue!50!black},
    citecolor={blue!50!black},
    urlcolor={blue!80!black}
}
\usepackage{pgfplots}
\usepackage{pgfplotstable}
\usepackage{babel}[english]
\usepackage{cancel}
%\usepackage[super]{natbib}
\pgfplotsset{compat=1.12}
%\usepackage{biblatex}
\usepackage[backend=bibtex, urldate=long,
    sorting=none,
    maxnames=10,
    defernumbers,style=numeric,autocite=superscript]{biblatex}
\addbibresource{sources.bib}
%\usepackage[fixlanguage]{babelbib}
%\DefineBibliographyStrings[english]{%
%    urlseen={last visited},
%}

\DeclareFieldFormat{urldate}{\mkbibparens{\bibstring{last visited on}\space#1}}


% rust formatting
\usepackage{color}
\usepackage{listings}
\usepackage{calc}
\usepackage{settobox}
%\usepackage{biblatex}
\definecolor{GrayCodeBlock}{RGB}{241,241,241}
\definecolor{BlackText}{RGB}{110,107,94}
\definecolor{RedTypename}{RGB}{182,86,17}
\definecolor{GreenString}{RGB}{96,172,57}
\definecolor{PurpleKeyword}{RGB}{184,84,212}
\definecolor{GrayComment}{RGB}{170,170,170}
\definecolor{GoldDocumentation}{RGB}{180,165,45}
\lstdefinelanguage{rust}
{
    columns=fullflexible,
    keepspaces=true,
    frame=single,
    framesep=0pt,
    framerule=0pt,
    framexleftmargin=4pt,
    framexrightmargin=4pt,
    framextopmargin=5pt,
    framexbottommargin=3pt,
    xleftmargin=4pt,
    xrightmargin=4pt,
%    numbers=left, % display line numbers on the left
    backgroundcolor=\color{GrayCodeBlock},
    basicstyle=\ttfamily\color{BlackText},
    keywords={
    true,false,
    unsafe,async,await,move,
    use,pub,crate,super,self,mod,
    struct,enum,fn,const,static,let,mut,ref,type,impl,dyn,trait,where,as,
    break,continue,if,else,while,for,loop,match,return,yield,in
},
    keywordstyle=\color{PurpleKeyword},
    ndkeywords={
    bool,u8,u16,u32,u64,u128,i8,i16,i32,i64,i128,char,str,
    Self,Option,Some,None,Result,Ok,Err,String,Box,Vec,Rc,Arc,Cell,RefCell,HashMap,BTreeMap,
    macro_rules
},
    ndkeywordstyle=\color{RedTypename},
    comment=[l][\color{GrayComment}\slshape]{//},
    morecomment=[s][\color{GrayComment}\slshape]{/*}{*/},
    morecomment=[l][\color{GoldDocumentation}\slshape]{///},
    morecomment=[s][\color{GoldDocumentation}\slshape]{/*!}{*/},
    morecomment=[l][\color{GoldDocumentation}\slshape]{//!},
    morecomment=[s][\color{RedTypename}]{\#![}{]},
    morecomment=[s][\color{RedTypename}]{\#[}{]},
    stringstyle=\color{GreenString},
    string=[b]"
}

\lstdefinelanguage{cpp}
{
    columns=fullflexible,
    keepspaces=true,
    frame=single,
    framesep=0pt,
    framerule=0pt,
    framexleftmargin=4pt,
    framexrightmargin=4pt,
    framextopmargin=5pt,
    framexbottommargin=3pt,
    xleftmargin=4pt,
    xrightmargin=4pt,
%    numbers=left, % display line numbers on the left
    backgroundcolor=\color{GrayCodeBlock},
    basicstyle=\ttfamily\color{BlackText},
    keywords={
    alignas,alignof,asm,auto,break,case,catch,char,char8_t,char16_t,char32_t,
    class,const,consteval,constexpr,constinit,continue,decltype,default,delete,do,
    double,dynamic_cast,else,enum,explicit,export,extern,false,float,for,
    friend,goto,if,inline,int,long,mutable,namespace,new,noexcept,nullptr,
    operator,private,protected,public,register,reinterpret_cast,return,
    short,signed,sizeof,static,static_assert,static_cast,struct,switch,
    template,this,thread_local,throw,true,try,typedef,typeid,typename,
    union,unsigned,using,virtual,void,volatile,wchar_t,while
},
    keywordstyle=\color{PurpleKeyword},
    ndkeywords={
    std,string,vector,map,set,unordered_map,unordered_set,shared_ptr,
    unique_ptr,weak_ptr,array,optional,variant,tuple,pair,ostream,
    istream,cout,cin,endl
},
    ndkeywordstyle=\color{RedTypename},
    comment=[l][\color{GrayComment}\slshape]{//},
    morecomment=[s][\color{GrayComment}\slshape]{/*}{*/},
    stringstyle=\color{GreenString},
    string=[b]"
}

% get name of current section
\makeatletter
\newcommand*{\currentname}{\@currentlabelname}
\makeatother


% get name of current chapter
\let\Chaptermark\chaptermark
\def\chaptermark#1{\def\Chaptername{#1}\Chaptermark{#1}}



\tikzstyle directed=[postaction={decorate,decoration={markings, % arrows on the field lines
mark=at position .1 with {\arrowreversed[scale=1.5]{stealth}},
mark=at position .9 with {\arrowreversed[scale=1.5]{stealth}}}}]
\tikzstyle tangent=[postaction={decorate,decoration={markings, % Tangent to the field line
mark=at position .7 with {\draw[ultra thick,stealth-,green!60!black,solid](-12pt,0)--(12pt,0)node[above]{$\vec{B}$};}}}]
\tikzstyle fLines=[thick,dashed,directed,tangent]
% tikz stuff for the flow charts
\tikzstyle{startstop} = [rectangle, rounded corners, minimum width=3cm, minimum height=1cm,text centered, draw=black, fill=red!30]
\tikzstyle{io} = [trapezium, trapezium left angle=70, trapezium right angle=110, minimum width=1cm, minimum height=1cm, text centered, draw=black, fill=blue!30]
\tikzstyle{process} = [rectangle, minimum width=3cm, minimum height=1cm, text centered, draw=black, fill=orange!30]
\tikzstyle{decision} = [diamond, minimum width=3cm, minimum height=1cm, text centered, draw=black, fill=green!30]
\tikzstyle{arrow} = [thick,->,>=stealth]

% #1 number of teeth
% #2 radius intern
% #3 radius extern
% #4 angle from start to end of the first arc
% #5 angle to decale the second arc from the first
\newcommand{\gear}[5]{%
    \foreach \i in {1,...,#1} {%
        [rotate=(\i-1)*360/#1]
        (0:#2) arc (0:#4:#2)
        [rounded corners=1.5pt]
        -- (#4+#5:#3)
        arc (#4+#5:360/#1-#5:#3)
        -- (360/#1:#2)
    }
}

\newcommand{\mycancel}[2]{\overset{#1}{\cancel{#2}}}
\newcommand{\cancelunder}[2]{\underset{#1}{\cancel{#2}}}
\renewcommand{\deg}[0]{\ensuremath{^\circ}}
\newcommand{\abs}[1]{\left|#1\right|}
\newcommand{\cfbox}[2]{%
    \colorlet{currentcolor}{.}%
    {\color{#1}%
    \noindent\fbox{\vbox{\hsize\dimexpr\hsize-2\fboxsep\relax
    #2
    }}}%
}
\newcommand{\todo}[1]{\begin{center}\cfbox{red}{\textcolor{red}{Todo: #1}}\end{center}}

\oddsidemargin=-1cm
\evensidemargin=-1cm
\voffset=-1.5cm
\textheight=26cm
\textwidth=18cm



\thispagestyle{fancy}

\fancyhf{} % clear all fields
\newcommand{\footl}{\Chaptername}
\newcommand{\bottompage}{-- \thepage\ --}

\fancypagestyle{plain}{%
    \fancyhf{}
    \fancyhead[L]{\rule[-2ex]{0pt}{2ex}\small Matura Project}
    \fancyhead[R]{\small Jan Wilhelm 6e}
    \fancyfoot[L]{\small \footl}
    \fancyfoot[C]{\bottompage}
    \fancyfoot[R]{\small December 17, 2024}
    \renewcommand{\headrulewidth}{0.2pt}
    \renewcommand{\footrulewidth}{0.2pt}}
\pagestyle{plain}


\title{
    \linespread{1.8}
    Jan Wilhelm, 6e\\
    \linespread{1}
    \Huge \textbf{Designing, assembling and programming a foosball goalkeeper robot}\\
    \linespread{1.8}
    \vspace{0.5cm}
    \begin{figure}[H]
        \centering
        \includegraphics[width=0.55\linewidth]{../photos/title}
        \caption*{}
        \label{fig:title}
    \end{figure}
    \vspace{-0.8cm}
    \Large
    \textbf{Maturitätsarbeit}\\
    Kantonsschule Hohe Promenade, Gymnasium, Zürich 2024/2025\\
    \textbf{Supervisor:} Clemens Pohle\\
    \textbf{Advisor:} Dr. Hugo Leonel Cabrera Cifuentes\\
    \linespread{1}}
% custom date 17.12.2024
\date{December 17, 2024}

%\linespread{1.8}


% Document
\begin{document}

    \makeatletter
    \voffset=-1.5cm
    \begin{titlepage}
        \thispagestyle{fancy}
        \renewcommand{\headrulewidth}{0pt}
        \renewcommand{\footrulewidth}{0pt}
        \lhead{\includegraphics[scale=0.5]{../photos/hopro_logo}}
        \rhead{}
        \rfoot{}
        \cfoot{} % this is to remove the page number
        \hbox{}\vfill
        \begin{center}
        {\LARGE\@title}
            \\[3em]
            {\large\@author}\\[1.75em]
            {\large\@date}
        \end{center}
        \vspace{3cm}\vfill
    \end{titlepage}
    \makeatother
    \voffset=-2.0cm
    \textheight=26.4cm

    % todo: main questions: We/I or unpersonal?
    % -> I always use I
%    \maketitle
%    \mewpage

    \shipout\null


    \renewcommand{\bottompage}{}
    \tableofcontents
    \newpage


    ~\\
    \section*{Abstract}
    \begin{abstract}
        Whether a machine can beat a human in a specific task is as old as the first machines.
        Machines have already surpassed humans in many areas, such as calculators, chess,
        and even driving cars.
        But what about foosball (table soccer)?
        This project demonstrates the development of a machine-controlled goalkeeper for foosball,
        capable of reliably stopping slow balls.
        While the goalkeeper occasionally struggles with the timing required for fast balls and shooting,
        the results serve as a proof of concept.
        The machine uses a camera to monitor the game from below and motors to move the goalkeeper.
        Although only basic AI was employed,
        the approach shows promise for extending similar designs to other players on the table.
        This work highlights the potential of automated systems in fast-paced, real-time scenarios,
        while acknowledging room for improvement to enhance performance and timing.
    \end{abstract}
    \vspace{1cm}


    \section*{Acknowledgements}
    I would like to express my heartfelt gratitude to the following individuals and organizations for their invaluable support and contributions to my project:
    \begin{itemize}
        \item My parents, for their financial and technical support and encouragement throughout this journey.
        \item Ilena Teng and her father Robert Teng, for their assistance in soldering the cables to the IR sensor and understanding how the sensor works.
        \item FabLab, for providing access to laser cutting, 3D printing, and CNC machining, and for the helpful guidance of the team members.
        \item Gabriel Schneider from ZHAW, for his expertise and support in the design process.
        \item Clemens Pohle, for his continuous supervision and insightful feedback, which greatly improved the quality of this documentation.
    \end{itemize}
    Thank you all for your generosity and dedication, which made this project possible.

    \vspace{1cm}
    \section*{Statement of Originality}
    Ich, Jan Wilhelm aus der Klasse 6e, erkläre hiermit, dass ich die vorliegende Arbeit "Designing, assembling and programming a foosball goalkeeper robot" selbständig und ohne Benützung anderer als der angegebenen Quellen oder Hilfsmittel verfasst bzw. gestaltet habe.
    \vspace{1cm}\\
    Ort, Datum: \hspace{6cm} Unterschrift:


    \chapter{Introduction}\label{ch:introduction}
    \renewcommand{\bottompage}{-- \thepage\ --}
    \setcounter{page}{1}
    \section{Motivation}\label{sec:motivation}


\section{Objective}\label{sec:objective}


\section{Calculations}\label{sec:calculations}
\subsection{Motors}\label{subsec:motors}

\subsubsection{Objectives}\label{subsubsec:objectives}
We want to be to stop a ball with a velocity of maximum $7m/s$.

\subsubsection{Calculations for the moving motor}\label{subsubsec:moving_motor}
The distance from the attacker to the goal is 0.4m.
That means we have stop the ball in
\begin{equation}
    \label{eq:stopping_time}
    t = \frac{s}{v} = \frac{0.4m}{7m/s} = 0.057s
\end{equation}
Which is not a lot of time to process the image and move the motors.
The goalkeeper has to travel a maximum distance of 0.2m in 0.057s, that means we have an average velocity of
\begin{equation}
    \label{eq:average_velocity}
    v = \frac{s}{t} = \frac{0.2m}{0.057s} = 3.5m/s
\end{equation}
Which means we need a top speed of at least $2\cdot3.5m/s=7m/s$, as the motors have to accelerate and decelerate.
We also need to be able to move the motors in 0.057s, which means we need a maximum acceleration of
\begin{equation}
    \label{eq:acceleration}
    a = \frac{v}{t} = \frac{7m/s}{0.057s} = 122.8m/s^2
\end{equation}
We assume that the weight of the tubes is $\sim 0.1kg$.
The torque needed to move the tubes is
\begin{equation}
    \label{eq:torque}
    \tau = F \cdot r = m \cdot a \cdot r = 0.1kg \cdot 122.8m/s^2 \cdot 0.083m \approx 1Nm
\end{equation}
assuming the radius of the gear is 8.3m.
And the required top RPM is
\begin{equation}
    \label{eq:top_rpm}
    \text{RPM} = \frac{v}{2\pi r} \cdot 60 = \frac{7m/s}{2\pi \cdot 0.083m} \cdot 60 \approx 800\text{RPM}
\end{equation}
A Motor that fits those requirements is the \textbf{PD42-3-1141} from \textbf{Trinamic}.

\subsubsection{Calculations for the rotating motor}\label{subsubsec:rotating_motor}
The ball has a mass of 17g.
We assume that we have an angle of 45\deg  to accelerate the ball.
\\
\begin{center}
    \begin{tikzpicture}
        \def \r {2}
        \def \rsmall {0.5}
        \def \angle {45}
        \draw (0,0) -- ({-sin(\angle / 2)*\r}, {-cos(\angle / 2)*\r});
        \draw (0,0) -- ({sin(\angle / 2)*\r}, {-cos(\angle / 2)*\r}) node[right, midway] {$r=70mm$};
        \draw ({-sin(\angle / 2)*\rsmall}, {-cos(\angle / 2)*\rsmall}) arc(270-\angle/2:270+\angle/2:\rsmall) node[midway,below] {$45\deg$};
        \draw[->] ({-sin(\angle / 2)*\r}, {-cos(\angle / 2)*\r}) arc(270-\angle/2:270+\angle/2:\r) node[midway,below] {$a$};
%        \draw () circle (\r);
    \end{tikzpicture}
\end{center}
%    \\
That means we have the distance $a$ to accelerate the ball.
\begin{equation}
    \label{eq:distance}
    a = r \cdot \frac{\pi}{4} = 70mm \cdot \frac{\pi}{4} = 55mm
\end{equation}
The goal is to shoot the ball back at a speed of 7m/s.
The time to accelerate the ball is
\begin{equation}
    \label{eq:time}
    t = \frac{2\pi\cdot r}{v} = \frac{2\pi\cdot 70mm}{7m/s} = \frac{\pi}{50} \approx 0.06s
\end{equation}
The maximum speed can easily be calculated with
\begin{equation}
    \label{eq:max_speed}
    \omega = \frac{60s/min}{t} = \frac{60s/min}{0.6s/rotation} = 1000\text{RPM}
\end{equation}
For the motor that just rotates the figure we can use a DC motor with gears and an encoder.
% Pololu 10:1 Metal Gearmotor 37Dx65L mm 12V with 64 CPR Encoder (Helical Pinion) 4758
A motor that fits those requirements is the \textbf{Pololu 10:1 Metal Gearmotor 37Dx65L mm 12V with 64 CPR Encoder (Helical Pinion) 4758}.

\subsection{Camera}\label{subsec:camera}

\subsubsection{Optics}\label{subsubsec:lens}

\paragraph{Lens Equation}\label{par:lens_equation}

The relationship between the object width (FoV), sensor width, and distance to the object is given by:
\begin{equation}
    \text{Object Width (FoV)} = \text{Sensor Width} \times \frac{\text{Distance to Object (d)}}{\text{Focal Length (f)}}\label{eq:lens_equation}
\end{equation}
In our case that is:
(We change the height of the table from 600mm to 800mm, so that the focal length doesn't have to be too small)
\begin{equation}
    \label{eq:object_width_general}
    \text{Object Width (FoV)} = 4.8\text{mm} \times \frac{800\text{mm}}{3.6\text{mm}} = 1066.6\text{mm}
\end{equation}
That's still not enough, we need an even smaller focal length.
We can use a 2.8mm lens, which gives us:
\begin{equation}
    \label{eq:object_width_2.8mm}
    \text{Object Width (FoV)} = 4.8\text{mm} \times \frac{800\text{mm}}{2.8\text{mm}} = 1371.4\text{mm}
\end{equation}
Which is enough, as the table is 1200mm wide.
We can even lower the height to 700mm:
\begin{equation}
    \label{eq:object_width_2.8mm_700mm}
    \text{Object Width (FoV)} = 4.8\text{mm} \times \frac{700\text{mm}}{2.8\text{mm}} = 1200\text{mm}
\end{equation}


    \chapter{Objectives}\label{ch:objectives}
    \section{Objective}\label{sec:objective}

The primary objective of this project is to develop a robotic goalkeeper capable of competing effectively with human players in foosball.
This project serves as a proof of concept, demonstrating the feasibility of building such a machine with high performance and precision.
Specifically, the robotic goalkeeper should be able to:

\begin{itemize}
    \item Defend against balls traveling at speeds of up to $\qty[per-mode=symbol]{7}{\m\per\s}$, matching the typical speeds achieved by skilled players during fast shots.
    \item Return the ball with a shooting speed of $\qty[per-mode=symbol]{7}{\m\per\s}$.
    \item Achieve a positional accuracy of $\qty[per-mode=symbol]{1}{\mm}$, comparable to the precision of standard stepper motors, including the mechanical components for player movement.
\end{itemize}



    \chapter{Calculations}\label{ch:calculations}
    \subsection{Motors}\label{subsec:motors}

\subsubsection{Objectives}\label{subsubsec:objectives}
We want to be to stop a ball with a velocity of maximum $7m/s$.

\subsubsection{Calculations for the moving motor}\label{subsubsec:moving_motor}
The distance from the attacker to the goal is 0.4m.
That means we have stop the ball in
\begin{equation}
    \label{eq:stopping_time}
    t = \frac{s}{v} = \frac{0.4m}{7m/s} = 0.057s
\end{equation}
Which is not a lot of time to process the image and move the motors.
The goalkeeper has to travel a maximum distance of 0.2m in 0.057s, that means we have an average velocity of
\begin{equation}
    \label{eq:average_velocity}
    v = \frac{s}{t} = \frac{0.2m}{0.057s} = 3.5m/s
\end{equation}
Which means we need a top speed of at least $2\cdot3.5m/s=7m/s$, as the motors have to accelerate and decelerate.
We also need to be able to move the motors in 0.057s, which means we need a maximum acceleration of
\begin{equation}
    \label{eq:acceleration}
    a = \frac{v}{t} = \frac{7m/s}{0.057s} = 122.8m/s^2
\end{equation}
We assume that the weight of the tubes is $\sim 0.1kg$.
The torque needed to move the tubes is
\begin{equation}
    \label{eq:torque}
    \tau = F \cdot r = m \cdot a \cdot r = 0.1kg \cdot 122.8m/s^2 \cdot 0.083m \approx 1Nm
\end{equation}
assuming the radius of the gear is 8.3m.
And the required top RPM is
\begin{equation}
    \label{eq:top_rpm}
    \text{RPM} = \frac{v}{2\pi r} \cdot 60 = \frac{7m/s}{2\pi \cdot 0.083m} \cdot 60 \approx 800\text{RPM}
\end{equation}
A Motor that fits those requirements is the \textbf{PD42-3-1141} from \textbf{Trinamic}.

\subsubsection{Calculations for the rotating motor}\label{subsubsec:rotating_motor}
The ball has a mass of 17g.
We assume that we have an angle of 45\deg  to accelerate the ball.
\\
\begin{center}
    \begin{tikzpicture}
        \def \r {2}
        \def \rsmall {0.5}
        \def \angle {45}
        \draw (0,0) -- ({-sin(\angle / 2)*\r}, {-cos(\angle / 2)*\r});
        \draw (0,0) -- ({sin(\angle / 2)*\r}, {-cos(\angle / 2)*\r}) node[right, midway] {$r=70mm$};
        \draw ({-sin(\angle / 2)*\rsmall}, {-cos(\angle / 2)*\rsmall}) arc(270-\angle/2:270+\angle/2:\rsmall) node[midway,below] {$45\deg$};
        \draw[->] ({-sin(\angle / 2)*\r}, {-cos(\angle / 2)*\r}) arc(270-\angle/2:270+\angle/2:\r) node[midway,below] {$a$};
%        \draw () circle (\r);
    \end{tikzpicture}
\end{center}
%    \\
That means we have the distance $a$ to accelerate the ball.
\begin{equation}
    \label{eq:distance}
    a = r \cdot \frac{\pi}{4} = 70mm \cdot \frac{\pi}{4} = 55mm
\end{equation}
The goal is to shoot the ball back at a speed of 7m/s.
The time to accelerate the ball is
\begin{equation}
    \label{eq:time}
    t = \frac{2\pi\cdot r}{v} = \frac{2\pi\cdot 70mm}{7m/s} = \frac{\pi}{50} \approx 0.06s
\end{equation}
The maximum speed can easily be calculated with
\begin{equation}
    \label{eq:max_speed}
    \omega = \frac{60s/min}{t} = \frac{60s/min}{0.6s/rotation} = 1000\text{RPM}
\end{equation}
For the motor that just rotates the figure we can use a DC motor with gears and an encoder.
% Pololu 10:1 Metal Gearmotor 37Dx65L mm 12V with 64 CPR Encoder (Helical Pinion) 4758
A motor that fits those requirements is the \textbf{Pololu 10:1 Metal Gearmotor 37Dx65L mm 12V with 64 CPR Encoder (Helical Pinion) 4758}.

\subsection{Camera}\label{subsec:camera}

\subsubsection{Optics}\label{subsubsec:lens}

\paragraph{Lens Equation}\label{par:lens_equation}

The relationship between the object width (FoV), sensor width, and distance to the object is given by:
\begin{equation}
    \text{Object Width (FoV)} = \text{Sensor Width} \times \frac{\text{Distance to Object (d)}}{\text{Focal Length (f)}}\label{eq:lens_equation}
\end{equation}
In our case that is:
(We change the height of the table from 600mm to 800mm, so that the focal length doesn't have to be too small)
\begin{equation}
    \label{eq:object_width_general}
    \text{Object Width (FoV)} = 4.8\text{mm} \times \frac{800\text{mm}}{3.6\text{mm}} = 1066.6\text{mm}
\end{equation}
That's still not enough, we need an even smaller focal length.
We can use a 2.8mm lens, which gives us:
\begin{equation}
    \label{eq:object_width_2.8mm}
    \text{Object Width (FoV)} = 4.8\text{mm} \times \frac{800\text{mm}}{2.8\text{mm}} = 1371.4\text{mm}
\end{equation}
Which is enough, as the table is 1200mm wide.
We can even lower the height to 700mm:
\begin{equation}
    \label{eq:object_width_2.8mm_700mm}
    \text{Object Width (FoV)} = 4.8\text{mm} \times \frac{700\text{mm}}{2.8\text{mm}} = 1200\text{mm}
\end{equation}


    \chapter{Mechanics}\label{ch:mechanics}
    \section{Design}\label{sec:design}
The Design has to as simple as possible, because the more parts there are the more things can go wrong.
% maybe insert quote from elon musk lol
The main problem that one encounters when designing such a system is:
As soon as the motor moves the tube, and with him the player, the other motor can not rotate the player anymore and vice versa.
To solve this problem we have to design two parts/systems that can solve this problem.
The first connects two tubes that can rotate freely but if one moves the other one moves with it.
The second one is the opposite, it connects two tubes that can move freely in the x direction but if one rotates the other one rotates with it.
Therefore we have two main parts, we called them: Turn Side and Move Side.
\subsection{Move Side}\label{subsec:move-side}
The Move Side is the part that moves the player in the x direction.
We use a combination of a gear and a gear rack to move the player, additionally we have the mechanism that connects the two tubes.
A render of the gears and the gear rack can be seen here:
\begin{figure}[H]
    \centering
    \includegraphics[height=7cm]{../photos/move_side_gear}
    \caption[moveside1]{Move side gear and gear rack}
    \label{fig:move_side_gear}
\end{figure}
% Todo: insert a picture of the meachaism that connects the tubes

\subsection{Turn Side}\label{subsec:turn-side}
The Turn Side is the part that rotates the player.
The turning motor is connected to a tube containing the smaller tube on which the player sits.
The larger tube contains a slit along its length, and the smaller tube has a pin that fits (a screw in this case) into the slit.
A render of the turning mechanism can be seen here:
\begin{figure}[H]
    \centering
    \includegraphics[height=7cm]{../photos/turn_side}
    \caption[turnside]{Turn side with motor and slit}
    \label{fig:turn_side2}
\end{figure}


    \chapter{Construction}\label{ch:construction}
    \section{General}\label{sec:general3}
The construction of the planned models from chapter~\ref{ch:mechanics} required the use of 3D-printers, laser-cutters and CNC-machines.
As I did not have those heavy machines at home, I went to the FabLab Zurich\autocite{fablab} where I could use those machines.

\subsection{Shoot Side manufacturing}\label{subsec:turn-side-manufacturing}
First I started with the move side and cut the holder, gear and gear rack from Plexiglas and printed the holder using the 3D-printers.
The first parts had a lot of issues regarding the spacing between the parts, they were either too loose or did not fit at all.
Then I printed the connecting part and mounted the inner part on the tube using a nail and a hole through the plastic and the aluminum tube and glowed the nail in place.

\subsection{Move Side manufacturing}\label{subsec:move-side-manufacturing}
The other side turned out to be simpler, because there were only two parts: The holder and the small connector, which slides on the motor shaft and into the larger tube.
After first using the wrong material for the 3D-printer, which resulted in a messy and unstable part, I successfully printed both parts.

\subsection{Camera Holder}\label{subsec:camera-holder}
The camera holder also consists of two parts: the wooden base and the plastic holder.
I wanted to use wood for the base, as printing a larger rectangle would take a long time and would be a waste of material.
So I used a Computer numerical controlled (CNC) machine to cut out a hole where the 3D-printed part fits into nicely.
The camera holder required two attempts as the first time I did not make a larger hole for air to flow through to cool the camera.
The second model even had a hole for a small fan.
However, after \("\)carefully\("\) (touching it after letting it run for 10 min) measuring the cooling performance of the camera without a fan I came to the conclusion that the fan wasn't necessary.
\begin{figure}[H]
    \centering
    \begin{subfigure}{.5\textwidth}
        \centering
        \includegraphics[width=.8\textwidth]{../photos/cam}
        \caption[cam]{Camera without holder}
        \label{fig:cam}
    \end{subfigure}%
    \begin{subfigure}{.5\textwidth}
        \centering
        \includegraphics[width=.8\textwidth]{../photos/camholder_render}
        \caption[camholder]{Render of the camera holder}
        \label{fig:camholder_render}
    \end{subfigure}
    \caption{Undistorted example image}
    \label{fig:camholder}
\end{figure}


    \chapter{Electronics}\label{ch:electronics}
    \section{General}\label{sec:general2}
To move the motors I need a controller that sends the right commands to the motors.
This is achieved by using an Arduino\autocite{arduino} and an arduino cnc shield\autocite{cnc-shield}, which with the help of the DRV8825\autocite{drv8825} driver controls the PD42-3-1141 stepper motor.
The Pololu DC motor is driven by the L298N\autocite{l298n} stepper motor driver, which is also able to run DC motors.
%\todo{Add eg circuit diagramm and/or irl image}
A schematic of the electronics can be seen here:

\begin{center}
    \begin{tikzpicture}[node distance=2cm]
        \node (in1) [io] {Camera};
        \node (pro1) [process, below of=in1] {Image processing};
        \node (pro2) [process, below of=pro1] {Ball detection and prediction};
        \node (pro3) [process, below of=pro2] {Arduino};
        \node (pro4) [process, below of=pro3, right of=pro3, xshift=1.5cm] {CNC shield};
        \node (pro5) [process, below of=pro4] {DRV8825};
        \node (out1) [io, below of=pro5] {Stepper Motor};
        \node (pro6) [process, below of=pro3, left of=pro3, xshift=-1.5cm] {L298N};
        \node (out2) [io, below of=pro6] {DC Motor};

        % arrows
        \draw [arrow] (in1) -- (pro1);
        \draw [arrow] (pro1) -- (pro2);
        \draw [arrow] (pro2) -- (pro3);
        \draw [arrow] (pro3) |- (pro4);
        \draw [arrow] (pro4) -- (pro5);
        \draw [arrow] (pro5) -- (out1);
        \draw [arrow] (pro3) |- (pro6);
        \draw [arrow] (pro6) -- (out2);
%        make box around pro1 and pro2 as they are on the computer
        \draw[red,thick] ($(pro1.north west)+(-1.3,0.6)$)  rectangle ($(pro2.south east)+(0.3,-0.3)$) node[right,midway, xshift=3cm] {Software (Chapter~\ref{ch:software})};
%       make box from pro6 to out1
        \draw[red,thick] ($(pro6.north west)+(-1,2.3)$)  rectangle ($(out1.south east)+(2.3,-0.3)$) node[below,midway, yshift=-4cm] {Electronics (Chapter~\ref{ch:electronics})};
    \end{tikzpicture}
\end{center}


\section{DC Motor}\label{sec:dc-motor}
A DC Motor has only two inputs and the direction is controlled by the polarity of the two inputs.
The speed can be controlled by only sending short pulses of power.
To read out the position of the motor and encoder is used.
In the encoder attachment at the end of the motor (the black part) contains a disk with small slits that rotates with the motor and a sensor (often optical) that recognizes when the slit passes.
The position can now be calculated by counting those passes, and because there are two sensors, the direction can be determined by looking at the phase difference.
A graph of the outputs can be seen here:
\begin{center}
    \begin{figure}[H]
        \centering
        \begin{subfigure}{.5\textwidth}
            \centering
            \begin{tikzpicture}
                \centering
                \begin{axis}
                    [
                    axis x line=center,
                    axis y line=center,
%                    width={0.4\linewidth},
%                    title={Encoder output while going forward\\},
                    xtick=none,
                    ytick={0,1},
                    yticklabels={low, high},
                    xlabel={$t$},
                    ylabel={$V$},
                    xlabel style={below right},
                    ylabel style={above left},
                    xmin=-2,
                    xmax=22,
                    ymin=-0.2,
                    ymax=1.2]

                    \addplot [blue,const plot] table {
                        0 0
                        1 1
                        4 0
                        7 1
                        10 0
                        13 1
                        16 0
                        19 1
                        23 0
                    };
                    \addplot [red, const plot] table {
                        0 0
                        2 1
                        5 0
                        8 1
                        11 0
                        14 1
                        17 0
                        20 1
                        23 0
                    };
                \end{axis}
            \end{tikzpicture}
            \caption[encoder-forwar]{Encoder output while going forward}
            \label{fig:encoder-forward}
        \end{subfigure}%
        \begin{subfigure}{.5\textwidth}
            \centering
            \begin{tikzpicture}
                \centering
                \begin{axis}
                    [
                    axis x line=center,
                    axis y line=center,
%                    width={0.4\linewidth},
%                    title={Encoder output while going forward\\},
                    xtick=none,
                    ytick={0,1},
                    yticklabels={low, high},
                    xlabel={$t$},
                    ylabel={$V$},
                    xlabel style={below right},
                    ylabel style={above left},
                    xmin=-2,
                    xmax=22,
                    ymin=-0.2,
                    ymax=1.2]

                    \addplot [blue,const plot] table {
                        0 0
                        1 1
                        4 0
                        7 1
                        10 0
                        13 1
                        16 0
                        19 1
                        23 0
                    };
                    \addplot [red, const plot] table {
                        0 0
                        0 1
                        3 0
                        6 1
                        9 0
                        12 1
                        15 0
                        18 1
                        21 0
                        23 0
                    };
                \end{axis}
            \end{tikzpicture}
            \caption[encoder-backward]{Encoder output while going backward}
            \label{fig:encoder-backward}
        \end{subfigure}
        \caption{Encoder outputs when the DC motor is turning}
        \label{fig:encoder}
    \end{figure}
\end{center}

An implementation in rust of a function that counts the passes can be seen here:
\begin{lstlisting}[language=rust,breaklines,label={lst:dc-motor}]
pub fn count_pos(&mut self, a_b: (bool, bool)) -> i32 {
    let a = a_b.0;
    let b = a_b.1;
    // check if something has changed
    if a != self.last_a || b != self.last_b {
        self.position += if (a as i32 - self.last_a as i32) == 0 { 1 * if a { 1 } else { -1 } * if (b as i32 - self.last_b as i32) == 1 { 1 } else { -1 } } else { - (1 * if b { 1 } else { -1 } * (a as i32 - self.last_a as i32))};
        self.last_a = a;
        self.last_b = b;
    }
    self.position
}

\end{lstlisting}


\section{Stepper Motor}\label{sec:stepper-motor}
The PD42-3-1141 stepper motor doesn't have an encoder, so the program first sets the position to 0 and then the motor can always do relatively N steps and recompute the position of the stepper motor.
One problem that can occur is that the motor skips steps, especially when sending the step command with a low delay in between.
To mitigate this problem, I let the motor accelerate, which is done by starting with a long delay and then lowering it.
Deceleration is achieved by just making the delay longer again.
The code for the stepper motor can be seen here:
\begin{lstlisting}[language=cpp,breaklines,label={lst:stepper-motor}]
void loop {
    // ... other logic code
    if (currentPosition != targetPosition) {
        int direction = (targetPosition > currentPosition) ? HIGH : LOW;
        // Set the direction of the motor
        digitalWrite(dirXPin, direction);

        int stepsToTarget = abs(targetPosition - currentPosition);

        if (stepsToTarget > (maxPulseWidthMicros - pulseWidthMicros) / acc) {
            // Accelerate to max speed
            pulseWidthMicros = max(minPulseWidthMicros, pulseWidthMicros - acc);
        } else {
            // Decelerate as the motor approaches the target
            pulseWidthMicros = min(maxPulseWidthMicros, pulseWidthMicros + acc);
        }

        // Enable the motor
        digitalWrite(enPin, LOW);
        // Step the motor for each half-step
        digitalWrite(stepXPin, HIGH);
        delayMicroseconds(pulseWidthMicros);
        digitalWrite(stepXPin, LOW);
        delayMicroseconds(pulseWidthMicros);

        currentPosition += (direction == HIGH) ? 1 : -1;
    } else {
        // Disable the motor
        digitalWrite(enPin, HIGH);
    }
}
\end{lstlisting}
Its important that the code first rounds the input to the next integer which is divisible by the step-resolution, in this case 4, because the motor can only stop at full steps.\\
\vspace{0.5cm}\\
The PD42-3-1141 stepper motor has 200 steps per rotation, that means the controller sends 200 step-commands until one full rotation is achieved.
But stepper motors can also do smaller steps, called half, 1/4, 1/8 etc. steps.
The smaller the steps, the higher the accuracy.
I chose 1/4 steps because the smaller the steps, the slower the motor and 1/4 seemed a good compromise.
But how can the arduino send the right commands to the motors?
First I tried to use the PCB that came with the motor, but I couldn't get it to work.
It just vibrated and didn't move.
So switched to the same driver that I use for the DC motor, because it's actually a stepper motor driver that can also drive DC motors.
That surprisingly worked, and I could control the motor.
Bu\(t\) I couldn't do micro steps as the LN298N driver doesn't support it.
After speaking with some colleagues, they suggested looking at the whole construction as a CNC machine, I know that sounds a little bit strange, but a CNC machine is technically also just some motors controlled by one or more controllers.
Therefore, I bought a CNC shield for the arduino, which is a shield that can control up to 4 stepper motors.
The shields don't come with drivers, so I had to buy them separately.
I didn't want to make the same mistake again, so I checked some reviews and found the DRV8825\autocite{drv8825} driver, which can do up to 1/32 steps and is also compatible with the CNC shield.
After some soldering and connecting the motors to the shield, I can now control the motors with the arduino.
The next problem with the stepper motor is that it sometimes skips steps.
Skipping one step though isn't a huge problem, but after some time the small errors add up and the motor is out of sync.
To mitigate this problem, I added an IR distance sensor to the motor, which measures the distance which the tube has traveled and then adjusts the position of the stepper motor.

\subsection{IR Distance Sensor}\label{subsec:ir-distance-sensor}
\todo{Not yet testet lol}

\subsection{Arduino code}\label{subsec:arduino-code}
Why do I use c++ for the code runing on the arduino?
Usually I want to write all my code in rust, as it is a memory safe language and has a lot of other advantages, because it is quite new and has a lot of modern features.
One feature is the ease of importing other libraries, which in rust are called crates, which is not as easy in c++.
There is an arduino framework for rust, but I had problems with the serial connection, I couldn't get it to work reliably.
Therefore, I switched back to the default arduino framework, which is written in c++.
At first, I had a complex encoding scheme, which encoded all commands in different bytecodes.
However, then I realised that it was just simpler to send an integer for the position and, for example, a capital \("\)R\("\) for the reset command.
Implementing that in c++ for the arduino was quite easy, as the arduino framework has a lot of built-in functions for serial communication.


    \chapter{Software}\label{ch:software}
    \section{General}\label{sec:general}
The software has the task to gather the image information from the camera, detecting the ball, predicting where it will go in the future and controlling the motors to either stop or shoot the ball.
This task can be split into the following chapters:
\begin{itemize}
    \item \textbf{Optics}: Correcting for Lens distortion
    \item \textbf{Ball Detection}: Detecting the ball in the image and converting the coordinates to real world coordinates
    \item \textbf{Prediction}: Predicting the ball movement
    \item \textbf{Controlling the motors}: Finally, we move the motors to the correct position
\end{itemize}


\section{Optics}\label{sec:optics}
The goal here is to correct for lens distortion, this can be achieved by first capturing many images containing a checkerboard pattern with known gird size, in this is case 8x8.
One such image in my case looks like this:
\begin{figure}[H]
    \centering
    \includegraphics[height=5cm]{../photos/calibration_image}
    \caption[calimage]{Calibration image}
    \label{fig:calibration_image}
\end{figure}
As one can see there is severe distortion in the image, this can be corrected by using the OpenCV library.

% todo: should i explain in detail how opencv does it?
% lets do it and later i can delete it if its too much

\subsection{OpenCV}\label{subsec:opencv}
OpenCV is a library that provides many functions for image processing, one of these functions is the camera calibration function.
The undistort function takes the images of the checkerboard pattern and calculates the distortion coefficients and the camera matrix.

The key principal is to map the distorted image to an undistorted image, this is done by calculating the pixel coordinates of the undistorted image for each pixel in the distorted image.
The Formulas for this are:
\begin{itemize}
    \item Radial distortion: This is the distortion that makes straight lines appear curved:
    \begin{gather*}
        r^2 = x^2 + y^2\\
        x_{\text{radial}} = x(1 + k_1 r^2 + k_2 r^4 + k_3 r^6)\\
        y_{\text{radial}} = y(1 + k_1 r^2 + k_2 r^4 + k_3 r^6)\\
    \end{gather*}
    \item Tangential distortion: This distortion occurs because the lens and the image sensor are not perfectly parallel.
    \begin{gather*}
        x_{\text{tangential}} = 2p_1 xy + p_2(r^2 + 2x^2)\\
        y_{\text{tangential}} = p_1(r^2 + 2y^2) + 2p_2 xy\\
    \end{gather*}
\end{itemize}
Together that gives:
\begin{gather*}
    x' = x_{\text{radial}} + x_{\text{tangential}}\\
    y' = y_{\text{radial}} + y_{\text{tangential}}\\
\end{gather*}
Finally, the undistorted pixel positions are transformed back to image coordinates using the camera matrix \( K \):
\[
    \begin{bmatrix}
        u \\
        v
    \end{bmatrix}
    =
    K \cdot
    \begin{bmatrix}
        x' \\
        y' \\
        1
    \end{bmatrix}
\]
where $K$ is the camera matrix:
\begin{center}
    K = \begin{bmatrix}
            f_x & 0   & c_x \\
            0   & f_y & c_y \\
            0   & 0   & 1
    \end{bmatrix}\\
    where:
\end{center}
\begin{itemize}
    \item $f_x$ and $f_y$ are the focal lengths along the x and y axes (in pixels),
    \item $c_x$ and $c_y$ are the coordinates of the optical center (principal point), typically at the center of the image.
\end{itemize}
These formulas allow the distorted image points to be remapped to undistorted coordinates.

\subsubsection{Implementation}\label{subsubsec:implementation}


Using the undistort function provided by opencv is not fast enough for our needs, because we only have 3ms to do the whole image processing and motor controlling.
So I wrote a custom function that generates a table where the index for each pixel in the new image is stored, so the function doesnt have to calculate the pixel coordinate for the undistorted image each time when the function is called.
The rust function to get the coordinates $x_{\text{original}},y_{\text{original}}$ of a pixel at $x_{\text{undistorted}},y_{\text{undistorted}}$ in the distorted (original) image looks like this:



\begin{lstlisting}[language=rust,breaklines,label={lst:distort_coords}]
/// Calculate the distorted coordinates given undistorted image coordinates.
fn distort_coords(x: f64, y: f64, fx: f64, fy: f64, cx: f64, cy: f64) -> (f64, f64) {
    // Distortion coefficients
    let k1 = DIST_COEFFS[0];
    let k2 = DIST_COEFFS[1];
    let p1 = DIST_COEFFS[2];
    let p2 = DIST_COEFFS[3];
    let k3 = DIST_COEFFS[4];

    // Normalize coordinates to [-1, 1]
    let x_normalized = (x - cx) / fx;
    let y_normalized = (y - cy) / fy;

    // Calculate radial distance
    let r2 = x_normalized * x_normalized + y_normalized * y_normalized;
    let r4 = r2 * r2;

    // Apply radial distortion
    let radial_distortion = 1.0 + k1 * r2 + k2 * r4 + k3 * r4 * r2;
    let x_radial = x_normalized * radial_distortion;
    let y_radial = y_normalized * radial_distortion;

    // Apply tangential distortion
    let x_tangential =
        2.0 * p1 * x_normalized * y_normalized + p2 * (r2 + 2.0 * x_normalized * x_normalized);
    let y_tangential =
        p1 * (r2 + 2.0 * y_normalized * y_normalized) + 2.0 * p2 * x_normalized * y_normalized;

    // Distorted coordinates
    let x_distorted = x_radial + x_tangential;
    let y_distorted = y_radial + y_tangential;

    // Map back to pixel coordinates
    let distorted_x = fx * x_distorted + cx;
    let distorted_y = fy * y_distorted + cy;

    (distorted_x, distorted_y)
}
\end{lstlisting}
Where \texttt{DIST\_COEFFS} are the distortion coefficients calculated by the OpenCV calibration function.
To generate the table I used the following code:
\begin{lstlisting}[language=rust,breaklines,label={lst:gen_table}]
/// Generate precomputation table for undistortion.
pub fn gen_table(
    original_width: u32, original_height: u32,
    new_width: u32, new_height: u32,
    x_offset: i32, y_offset: i32,
) -> Vec<i32> {
    // Camera matrix values
    let fx = CAMERA_MATRIX[0][0];
    let fy = CAMERA_MATRIX[1][1];
    let cx = CAMERA_MATRIX[0][2];
    let cy = CAMERA_MATRIX[1][2];
    let mut precomputation_table = vec![];

    for y in 0..new_height {
        for x in 0..new_width {
            let x = x as i32 + x_offset;
            let y = y as i32 + y_offset;
            // Map the pixel back to the distorted image coordinates
            let (distorted_x, distorted_y) = distort_coords(x as f64, y as f64, fx, fy, cx, cy);

            let distorted_x = distorted_x.round() as i32;
            let distorted_y = distorted_y.round() as i32;

            // If the coordinates are within the bounds of the original image, map the pixel
            let index = if distorted_x >= 0
                && distorted_x < original_width as i32
                && distorted_y >= 0
                && distorted_y < original_height as i32
            {
                let index = ((distorted_y * original_width as i32 + distorted_x) * 3) as usize;
                index as i32
            } else {
                -1 // black pixel (outside the image bounds)
            };
            precomputation_table.push(index);
        }
    }
    precomputation_table
}
\end{lstlisting}
This generates a long vector with the corresponding index for each pixel in the undistorted image.
(Note that these are indices because the image is flattened to a vector of length $3 \cdot \text{width} \cdot \text{height}$, where each pixel has 3 values for the RGB channels.)

Using the table is as simple as just looking up the index for the pixel in the undistorted image and copying the pixel values from the original image to the new image.
This function is called for each frame and the result is a corrected image with no distortion.
The Code can be seen here:
\begin{lstlisting}[language=rust,breaklines,label={lst:undistort_image_table}]
/// Undistort an image using the precomputed table.
pub fn undistort_image_table(
    img: &[u8],
    undistorted_img: &mut [u8],
    table: &Vec<i32>,
    new_width: u32,
    new_height: u32,
) {
    // Assert that the image has the correct size
    assert_eq!(
        undistorted_img.len(),
        3 * new_width as usize * new_height as usize
    );

    for i in 0..new_height * new_width {
        let index = table[i as usize];

        if index != -1 {
            #[inline]
            /// Helper function to avoid code duplication
            fn set_pixel(undistorted_img: &mut [u8], img: &[u8], pixel_index: usize, i: usize) {
                undistorted_img[i as usize] = img[pixel_index];
            }
            let pixel_index = index as usize;
            set_pixel(undistorted_img, img, pixel_index, i as usize * 3);
            set_pixel(undistorted_img, img, pixel_index + 1, i as usize * 3 + 1);
            set_pixel(undistorted_img, img, pixel_index + 2, i as usize * 3 + 2);
        }
    }
}
\end{lstlisting}
The parameters for this function are a bit special because one argument is the final vector where the undistorted image is stored, the other is the original image, and the last one is the table that was generated before.
Giving the final image buffer as a parameter saves time, because the image buffer can be reused in each frame, saving the time to reallocate the buffer each frame.
\\
An illustration can be seen in this example rainbow image being given to the undistort function:
% todo: pick a better example image
\begin{figure}[H]
    \centering
    \begin{subfigure}{.5\textwidth}
        \centering
        \includegraphics[width=.8\textwidth]{../photos/original_rainbow}
        \caption[originalRainbow]{Original rainbow image}
        \label{fig:original_rainbow}
    \end{subfigure}%
    \begin{subfigure}{.5\textwidth}
        \centering
        \includegraphics[width=.8\textwidth]{../photos/undistorted_rainbow}
        \caption[originalRainbow]{Undistorted rainbow image}
        \label{fig:undistorted_rainbow}
    \end{subfigure}
    \caption{Undistorted rainbow image}
    \label{fig:original_undistorted_rainbow}
\end{figure}
The undistorted image\ref{fig:undistorted_rainbow} is larger than the original rainbow image\ref{fig:original_rainbow}, that is because some pixel are moved out of the original image bounds because the camera can see \"further\" at the corners than at the sides.
This effect can also be seen in some example footage of the table:
\begin{figure}[H]
    \centering
    \begin{subfigure}{.5\textwidth}
        \centering
        \includegraphics[width=.8\textwidth]{../photos/original_example8}
        \caption[originalRainbow]{Original image}
        \label{fig:original_example8}
    \end{subfigure}%
    \begin{subfigure}{.5\textwidth}
        \centering
        \includegraphics[width=.8\textwidth]{../photos/output8}
        \caption[originalRainbow]{Undistorted image}
        \label{fig:undistorted_example8}
    \end{subfigure}
    \caption{Undistorted example image}
    \label{fig:original_undistorted_example}
\end{figure}
Therefore we have to crop the image at the sides by a different amount, I made a simple graphical user interface (GUI) to adjust the cropping values.
The GUI can be seen here:
\begin{figure}[H]
    \centering
    \includegraphics[width=0.8\textwidth]{../photos/margin_adj_gui}
    \caption[marginadjgui]{Margin Adjustment GUI}
    \label{fig:margin_adj_gui}
\end{figure}
The GUI is also written in rust using the egui\autocite{egui} library.
The cropped example image looks like this:
\begin{figure}[H]
    \centering
    \includegraphics[width=0.8\textwidth]{../photos/example8_cropped}
    \caption[croppedexampleimage]{Cropped example image}
    \label{fig:example8_cropped}
\end{figure}

\subsection{Real World Coordinates}\label{subsec:real-world-coordinates}
An important part of the software is to convert the pixel coordinates to real-world coordinates.
This is done by adding fiducials (markers that are known to be at a specific position) to the table.
The fiducials are placed at the corners of the table, and the coordinates are measured in millimeters.
They can be seen here:
\begin{figure}[H]
    \centering
    \begin{subfigure}{.2\textwidth}
        \centering
        \fbox{\includegraphics[width=.7\textwidth]{../photos/fiducials/fiducial_1}}
        \caption[originalRainbow]{Fiducial 1}
        \label{fig:fid_1}
    \end{subfigure}%
    \begin{subfigure}{.2\textwidth}
        \centering
        \fbox{\includegraphics[width=.7\textwidth]{../photos/fiducials/fiducial_2}}
        \caption[originalRainbow]{Fiducial 2}
        \label{fig:fid_2}
    \end{subfigure}
    \begin{subfigure}{.2\textwidth}
        \centering
        \fbox{\includegraphics[width=.7\textwidth]{../photos/fiducials/fiducial_3}}
        \caption[originalRainbow]{Fiducial 3}
        \label{fig:fid_3}
    \end{subfigure}
    \begin{subfigure}{.2\textwidth}
        \centering
        \fbox{\includegraphics[width=.7\textwidth]{../photos/fiducials/fiducial_4}}
        \caption[originalRainbow]{Fiducial 4}
        \label{fig:fid_4}
    \end{subfigure}
    \caption{Fiducials}
    \label{fig:fiducials_all}
\end{figure}
To calibrate the real world coordinates one has to detect the fiducials reliably in the image.
This can be done by using a convolutional neural network (CNN) to detect the fiducials.

\subsubsection{Training Data}\label{subsubsec:training-data}
To train the CNN, one needs a lot of images containing the fiducials, making such a lot of images and then checking where exactly the coordinates of the midpoint of the fiducial lies is very tedious and not practical.
Therefore i wrote a Java programm to generate about 250 labeled images to train the CNN.
An image containing all the fiducials looks like this (the red point shows the midpoint of the fiducial):
\begin{figure}[H]
    \centering
    \includegraphics[width=0.8\textwidth]{../photos/training_whole_general_image}
    \caption[fiducials]{Training fiducials}
    \label{fig:fiducials}
\end{figure}
The trick I am using is that I know that the fiducials always lie in the corners of the table, so I can generate the images by just placing the fiducials in the corners with some random offset.
The different parts can be seen here:

\begin{figure}[H]
    \centering
    \begin{tikzpicture}
        % Include the image
        \node[anchor=south west, inner sep=0] (image) at (0,0) {\includegraphics[width=0.8\textwidth]{../photos/base_img}};
        \begin{scope}[x={(image.south east)}, y={(image.north west)}]
            % vertical lines (6 parts)
            \foreach \i in {1/6, 2/6, 3/6, 4/6, 5/6} {
                \draw[red, very thick] (\i, 0) -- (\i, 1);
            }
            % horizontall lines (4 parts)
            \foreach \i in {1/4, 2/4, 3/4} {
                \draw[red, very thick] (0, \i) -- (1, \i);
            }
        \end{scope}
    \end{tikzpicture}
    \caption{Image split into 4 horizontal and 6 vertical parts}\label{fig:figure}
\end{figure}
\textbf{Create and Train the CNN}\\
I made two separate CNNs, one for detecting the fiducial (number from 1-4 as there are four fiducials) and one for detecting the coordinate of the midpoint of the fiducial (from where the real-world measurements are made).
The CNNs are trained using the tensorflow library made by google.
The structure of the CNNs can be seen here:
\begin{figure}[H]
    \centering
    \begin{subfigure}{.5\textwidth}
        \centering
        \includegraphics[width=.8\textwidth]{../photos/fiducial_classifier_model}
        \caption[originalRainbow]{Fiducial classifier model}
        \label{fig:fiducial_classifier_model}
    \end{subfigure}%
    \begin{subfigure}{.5\textwidth}
        \centering
        \includegraphics[width=.8\textwidth]{../photos/fiducial_coords_model}
        \caption[originalRainbow]{Fiducial coordinates model}
        \label{fig:fiducial_coords_model}
    \end{subfigure}
    \caption{Structure of the CNNs}
    \label{fig:CNN_structure}
\end{figure}

% Todo add stats about the real world training/tests
% Todo add heat map of gradient of the image (grad cam)



\section{Ball Detection}\label{sec:ball-detection}


\section{Prediction}\label{sec:prediction}


\section{Controlling the motors}\label{sec:controlling-the-motors}
To control the motors I use an Arduino Mega 2560, because it has enough pins to control all the motors at once.
The Arduino is connected to the computer via USB, and the computer sends the motor positions to the Arduino via serial communication.

\subsubsection{Controlling the DC-Motor}\label{subsubsec:controlling-the-dc-motor}
The DC-Motor is controlled by the stepper motor driver L298N as seen here\ref{ch:electronics}.






    \chapter{Results}\label{ch:results}
%    \todo{Finish project to get latest results}
    Describing the results is not an easy task, as the project is still in development and the final results are not yet available.
However, some general results can be presented, such as the construction of the models, the embedded programming, and the image processing and optics.

\subsection{Embedded Programming}\label{subsec:results_embedded}
The code on the Arduino is not as sophisticated as desired.
While it performs the tasks well enough at low speeds, at higher speeds the motor becomes inconsistent and inaccurate.
This issue can be resolved with the custom PCB provided by the company, which uses the RS485 communication standard.
Therefore, this improvement is listed as a future enhancement in Section~\ref{sec:improvements}.

\subsubsection{Precision}\label{subsubsec:precision}
The precision has not yet reached $1\,\mathrm{mm}$.
Measuring it precisely is challenging because the closer the player is, the more precise the IR sensor needs to be.
Additionally, the IR sensor occasionally fails to measure the distance correctly, leading to improper movement of the player.

\subsection{Image Processing and Optics}\label{subsec:results_image}
The undistortion function works very well, largely due to over 20 iterations of the calibration process.
Ball detection is also highly accurate, successfully detecting only the ball while avoiding the tube or the player.
However, under certain lighting conditions, the player’s hand is occasionally detected as the ball.
This issue could be resolved by using a differently colored ball.

The ball prediction is accurate, with a small margin of error.
However, timing inconsistencies remain, where the player sometimes shoots too early or too late.
This issue is attributed to camera latency and USB driver delays.
Additionally, the inexpensive physical DC motor driver used is outdated, unable to ramp up speed quickly enough, and prone to overheating.

\subsection{General}\label{subsec:general}
Overall, this project successfully demonstrates a working proof of concept for a foosball robot.
Its main advantage over similar projects lies in the simplicity of its design, which eliminates the need for motor movement.
This simplicity allows for easy expansion to heavier, faster, and stronger motors.




    \chapter{Conclusion}\label{ch:conclusion}
    \section{What I learned}\label{sec:learned}

\subsection{Hardware}\label{subsec:hardware}
I learned a lot of things while building the parts for the table regarding Computer assisted Design (CAD) and the process to physically build the parts.

\subsubsection{CAD}
%\todo{cite fusion}
After extended research, I found the CAD program Fusion 360\autocite{fusion360} made by Autodesk, which I could use to the full extend with a free education license\autocite{autodesk-education}.
Fusion 360 has a steep learning curve, meaning it took a lot of time at the beginning to design simple parts, but after having learned the basics by watching YouTube\autocite{youtube,fusion360-tutorial} tutorials, a lot of things made much more sense.
Also, the people at the FabLab\autocite{fablab} were very helpful, showing me special features in Fusion 360\autocite{fusion360}.

\subsubsection{Process}
I made most of the parts at the FabLab and that would not have been possible without the help of the people there, which showed me for example printer settings for the 3D printers or the laser cutter.
The CNC Milling course at the FabLab Zürich, where I learned the basics regarding CNC Milling, was also very helpful.

\subsection{Software}\label{subsec:software}
Regarding software I also learned a lot of new things, the main things being:

\subsubsection{Embedded Programming}\label{subsubsec:embedded}
Embedded programs are programms that run on computers without an operating system (OS), such as, for example, Windows\autocite{windows}, MacOS\autocite{macos} or Linux-based systems\autocite{linux}.\autocite{embedded-software}
I wrote embedded code for the Arduino, first in Rust\autocite{rust}, using the avrdude library\autocite{avrdude}, and then with Arduino C++\autocite{arduino-c++}, which was also a fascinating experience, as many features are not supported by embedded systems.
For example there are no files or folders, but also in Rust there are no floating point numbers, which means I had to write a custom library that supported fixed point numbers for my Proportional–integral–derivative controller (PID)\autocite{pid} controller for the DC motor
Floating point numbers\autocite{floating-point} is the standard way to represent non-integers, fixed point numbers\autocite{fixed-point} do that too, but have often less accuracy, but are faster to work with, if the processor does not natively support floating point numbers.

\subsubsection{Image processing and optics}
By implementing the undistortion function (cf. subsection~\ref{subsec:opencv}), I learned a lot about optics and how to efficiently perform an algorithm by precomputing as much as possible.
Also the ball detection (cf. subsection~\ref{sec:ball-detection}) provided some challenges, requiring thinking outside the box, for example the fact that it would recognize the tube as the ball, which I solved by implementing an algorithm that looks if the found shape is similar to a circle and has the correct radius.

\subsubsection{CNN}
Some time ago I implemented a Neural network (NN) from scratch in rust and once in java.
However, they were not very fast, as they did not use the graphical processing unit (gpu), which would be able to compute many operations in parallel.
Therefore, for this project, I used Tensorflow (TF), which was not always easy to use, but a lot faster than my own implementation.
While using TF, I learned a lot of new things about NNs, especially the different types of layers than can be used, and how to prevent a model from overfitting, meaning it just learns the train dataset by heart, instead of learning the connection between the data and the target output.
This is, for example, done by introducing dropout layers, which just deletes some of the internal data in the NN.
While it seems very counterproductive to delete perhaps important data, in the end prevents the NN from learning everything by heart, because each time random elements/numbers are deleted, so each time the input is a little bit different.


\section{What I would do differently next time}\label{sec:different}
I had a lot of problems with the stepper motor, so it would have been better to look for a stepper motor that definitely works with the arduino and look for a simple tutorial on how to use the motor.
Also writing the code for the stepper motor by myself was a more challenging task than expected, so it would have been better to use preexisting code, writing by people that have more knowledge than me.
Currently, my cable management is a huge mess, so next time, I would first of all color-code the cables in a way that makes sense, and then use cable management tools to keep the cables in place.
Furthermore, I would do more connections on a breadboard, as it is easier to change the connections, and then solder the connections, as it is more stable.


\section{Future improvements}\label{sec:improvements}
I suggest the following improvements:
\begin{itemize}
    \item A new way to control the stepper motor, the code written by me that accelerates and decelerates the motor is good enough for slow movements but for fast movements its not sophisticated enough.
    Therefore, I would like to use code specifically made for this stepper motor, which the company has already provided with the motor with a custom PCB at the back.
    Currently, I was not able to use that circuit board, because I was not aware of the fact that the arduino, with a additional breakout board, supports the RS458 communication standard.
    \item I would like to extend the table by adding the missing players in the team, by manufacturing the same parts again three times.
    A small problem is the fact that the defender travels more than half the width of the table, meaning I wont be able to connect the shoot motor in the same way as the player.
    To solve this issue, I will build an extended holder offset to the side, providing the space for the longer tube.
    \item Currently the camera captures the whole table with a framerate of $149$ fps, which is sufficient for slow balls, but the accuracy could be greatly improved by only capturing a small rectangle of the table, but at a higher framerate.
    This frame will then be moved around, according to the last position and the current velocity of the ball.
\end{itemize}


\section{Acknowledgements}\label{sec:acknowledgements}
I would like to express my heartfelt gratitude to the following individuals and organizations for their invaluable support and contributions to my project:
\begin{itemize}
    \item My parents, for their financial support and encouragement throughout this journey.
    \item Ilena Teng, for her assistance in soldering the cables to the IR sensor.
    \item FabLab, for providing access to laser cutting, 3D printing, and CNC machining, and for the helpful guidance of the team members.
    \item Gabriel from ZHAW, for his expertise and support in the design process.
    \item Mr. Pohle, for his insightful feedback, which greatly improved the quality of this documentation.
\end{itemize}
Thank you all for your generosity and dedication, which made this project possible.











    \newpage
    \renewcommand{\footl}{References}

    \bibliographystyle{abbrv}
    \printbibliography

%    \newpage
%    \bibliographystyle{unsrt}
%%    \renewcommand\refname{Quellen}
%    \makeatletter
%    \renewcommand\@biblabel[1]{\textbullet}
%    \makeatother
%    \bibliography{sources} % Entries are in the refs.bib file
%    \printbibliography
%    \nocite{*}

    \newpage


    \chapter{Appendix}\label{ch:appendix}
    \renewcommand{\footl}{\Chaptername}
    \section{Whole Flowchart}\label{sec:whole-flowchart}

\begin{center}
    \begin{tikzpicture}[node distance=2cm]
        \node (cam) [io] {Camera};
        \node (undistortion) [process, below of=cam] {Undistortion};
        \node (ball_detection) [process, below of=undistortion] {Ball detection};
        \node (prediction) [process, below of=ball_detection] {Prediction};
        \node (where) [process, below of=prediction, right of=prediction, xshift=1.5cm] {Where?};
        \node (when) [process, below of=prediction, left of=prediction, xshift=-1.5cm] {When?};
        \node (arduino) [process, below of=when, right of=when, xshift=1.5cm] {Arduino};
        \node (cnc) [process, below of=arduino, right of=arduino, xshift=1.5cm] {CNC shield};
        \node (drv) [process, below of=cnc] {DRV8825};
        \node (stepper) [io, below of=drv] {Stepper Motor};
        \node (l298n) [process, below of=arduino, left of=arduino, xshift=-1.5cm] {L298N};
        \node (dc) [io, below of=l298n] {DC Motor};

        % arrows
        \draw [arrow] (cam) -- (undistortion);
        \draw [arrow] (undistortion) -- (ball_detection);
        \draw [arrow] (ball_detection) -- (prediction);
        \draw [arrow] (prediction) |- (where);
        \draw [arrow] (prediction) |- (when);
        \draw [arrow] (where) |- (arduino) node[below,midway] {Target player position};
        \draw [arrow] (when) |- (arduino) node[below,midway] {Shoot the ball};

        \draw [arrow] (arduino) |- (cnc);
        \draw [arrow] (cnc) -- (drv);
        \draw [arrow] (drv) -- (stepper);
        \draw [arrow] (arduino) |- (l298n);
        \draw [arrow] (l298n) -- (dc);


        \coordinate (corner1) at ([xshift=-0.3cm]when.west |- undistortion.north);


%%        make box around undistortion and ball_detection as they are on the computer
        \draw[red,thick] ($(where.south east)+(0.3,-0.3)$) rectangle ([xshift=-0.3cm, yshift=0.3cm]when.west |- undistortion.north) node[above, xshift=2cm] {Software (Chapter~\ref{ch:software})};
%%       make box from arduino to stepper
        \draw[red,thick] ([xshift=-0.9cm, yshift=0.3cm]l298n.west |- arduino.north)  rectangle ($(stepper.south east)+(2.3,-0.3)$) node[below,midway, yshift=-4cm] {Electronics (Chapter~\ref{ch:electronics})};
    \end{tikzpicture}
\end{center}


\end{document}

% pdflatex -interaction=nonstopmode -output-directory=output src/main.tex

