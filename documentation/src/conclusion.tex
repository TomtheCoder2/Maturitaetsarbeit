\section{What I learned}\label{sec:learned}

\subsection{General}\label{subsec:general_learned}
Through this project, I gained knowledge across a broad range of topics, including hardware (mechanics and electronic components), software, mathematical theory, machine learning, and optics.
These fields are deeply interconnected, which enhanced my understanding of how diverse systems work together.\\
The project was highly motivating but also posed challenges, especially in sourcing components.
Deciding which parts to use, finding reliable suppliers, and managing delivery times required significant effort.
Additionally, reading technical documentation was often difficult, as it was unclear or lacked context, requiring me to invest time to fully understand it.\\
A key factor in the project’s success was the initial design phase.
I dedicated substantial effort, with valuable input and assistance, to creating a robust and well-thought-out design from the beginning, which meant that no major changes were necessary later.
This confirmed the value of thorough planning and encouraged me to approach future projects with similar diligence.\\
Overall, the positive experience further solidified my interest in pursuing mechanical engineering as a field of study, where I can continue to apply and expand the skills I developed during this project.

\subsection{Hardware}\label{subsec:hardware}
I learned a lot of things while building the parts for the table regarding Computer assisted Design (CAD) and the process to physically build the parts.

\subsubsection{CAD}
%\todo{cite fusion}
After extended research, I found the CAD program Fusion 360\autocite{fusion360} made by Autodesk, which I could use to the full extend with a free education license\autocite{autodesk-education}.
Fusion 360 has a steep learning curve, meaning it took me a lot of time at the beginning to design simple parts, but after having learned the basics by watching YouTube\autocite{youtube,fusion360-tutorial} tutorials, a lot of things made much more sense.
Also, the people at the FabLab\autocite{fablab} were very helpful, showing me special features in Fusion 360\autocite{fusion360}.

\subsubsection{Process}
I made most of the parts at the FabLab and that would not have been possible without the help of the people there, which showed me for example printer settings for the 3D printers or the laser cutter.
The CNC Milling course at the FabLab Zürich, where I learned the basics regarding CNC Milling, was also very helpful.

\subsection{Software}\label{subsec:software}
Regarding software I also learned a lot of new things, the main things being:

\subsubsection{Embedded Programming}\label{subsubsec:embedded}
Embedded programs are designed to run on computers that lack an operating system (OS) such as Windows\autocite{windows}, macOS\autocite{macos}, or Linux-based systems\autocite{linux}.\autocite{embedded-software}\\
For this project, I developed embedded code for the Arduino, initially using Rust\autocite{rust} with the avrdude library\autocite{avrdude}, and later with Arduino C++\autocite{arduino-c++}.
Both approaches provided unique challenges and learning opportunities, especially since many features commonly available in high-level systems are absent in embedded environments.\\
For example, embedded systems typically lack files and folders.
Additionally, in Rust, floating-point numbers are not natively supported on some microcontrollers.
To address this limitation, I implemented a custom library for fixed-point arithmetic, which I used to build a Proportional–Integral–Derivative (PID) controller\autocite{pid} for the DC motor.\\
While floating-point numbers\autocite{floating-point} are the standard for representing non-integers, fixed-point numbers\autocite{fixed-point} offer an alternative that, while often less precise, is computationally faster on processors without native floating-point support.

\subsubsection{Image processing and optics}
By implementing the undistortion function (cf.
subsection~\ref{subsec:opencv}), I learned a lot about optics and how to efficiently perform an algorithm by precomputing as much as possible.
Also the ball detection (cf.
subsection~\ref{sec:ball-detection}) provided some challenges, requiring thinking outside the box, for example the fact that it would recognize the tube as the ball, which I solved by implementing an algorithm that looks if the found shape is similar to a circle and has the correct radius.

\subsubsection{CNN}
Some time ago I implemented a neural network (NN) from scratch in rust and once in java.
However, they were not very fast, as they did not use the graphical processing unit (gpu), which would be able to compute many operations in parallel.
Therefore, for this project, I used Tensorflow (TF), which was not always easy to use, but a lot faster than my own implementation.
While using TF, I learned a lot of new things about NNs, especially the different types of layers than can be used, and how to prevent a model from overfitting, meaning it just learns the train dataset by heart, instead of learning the connection between the data and the target output.
This is, for example, done by introducing dropout layers, which just deletes some of the internal data in the NN.
While it seems very counterproductive to delete perhaps important data, in the end prevents the NN from learning everything by heart, because each time random elements/numbers are deleted, so each time the input is a little bit different.


\section{What I would do differently next time}\label{sec:different}
Reflecting on this project, there are several things I would approach differently in the future.\\
I encountered significant issues with the stepper motor, so next time, I would carefully research stepper motors to find one that is guaranteed to work seamlessly with the Arduino.
Additionally, I would look for a simple tutorial to better understand how to use the motor.
Writing the code for the stepper motor from scratch turned out to be more challenging than anticipated, so I would use preexisting libraries or code written by experienced developers to save time and avoid unnecessary difficulties.\\
Another improvement would be to plan my cable management more thoughtfully.
Currently, the cables are a mess, so I would color-code them logically and use cable management tools to keep everything organized.
For prototyping, I would do more connections on a breadboard initially, as it allows for easier adjustments, and then solder the connections for long-term stability.\\
I also realized the importance of capturing more photos during the construction process to document progress and challenges.
Moreover, I would spend more time considering which parts to use early on, as long delivery times significantly impacted my workflow.
Ordering all the required components at once, where possible, would help streamline the process and reduce delays.\\
Finally, I underestimated the complexity of the electronics, which required much more time and effort than expected.
In future projects, I would allocate more resources and planning to this aspect to ensure smoother progress.


\section{Future improvements}\label{sec:improvements}
I suggest the following improvements:
\subsection*{Controlling the motors}
A new way to control the stepper motor, the code written by me that accelerates and decelerates the motor is good enough for slow movements but for fast movements its not sophisticated enough.
Therefore, I would like to use code specifically made for this stepper motor, which the company has already provided with the motor with a custom PCB at the back.
Currently, I was not able to use that circuit board, because I was not aware of the fact that the arduino, with a additional breakout board, supports the RS458 communication standard.
\subsection*{Adding switches}
I would like to add switches to the tube, so that the motor knows when it reaches the end.
This would make the calibration process much easier, as the motor would just have to move the tube until it reaches the switch.
Furthermore, it would provide a safety feature, as the motor would stop moving if the tube is blocked.
\subsection*{Adding the missing players}
I would like to extend the table by adding the missing players in the team, by manufacturing the same parts again three times.
A small problem is the fact that the defender travels more than half the width of the table, meaning I wont be able to connect the shoot motor in the same way as the player.
To solve this issue, I will build an extended holder offset to the side, providing the space for the longer tube.
\subsection*{Camera frame rate}
Currently the camera captures the whole table with a framerate of $149$ fps, which is sufficient for slow balls, but the accuracy could be greatly improved by only capturing a small rectangle of the table, but at a higher framerate.
This frame will then be moved around, according to the last position and the current velocity of the ball.
\subsection*{Speed measurement}
The speed of the ball could be measured by the camera, which would could be converted to $m/s$ and shown on a display.
This feature could be used to measure the speed of the ball, and to adjust the speed of the player accordingly.

