\section{General}\label{sec:general3}
To construction of the planned models from chapter~\ref{ch:mechanics} required the use of 3D-printers, laser-cutters and CNC-machines.
As I didn't have those heavy machines at home, I went to the FabLab Zurich\autocite{fablab} where I could use those machines.

\subsection{Turn Side manufacturing}\label{subsec:turn-side-manufacturing}
First I started with the move side and cut the holder, gear and gear rack from Plexiglas and printed the holder using the 3D-printers.
The first parts had a lot of issues regarding the spacing between the parts, they were either too loose or did not fit at all.
Then I printed the connecting part and mounted the inner part on the tube using a nail and a hole through the plastic and the aluminum tube and glowed the nail in place.

\subsection{Move Side manufacturing}\label{subsec:move-side-manufacturing}
The other side turned out to be simpler, because there were only two parts: The holder and the small connector, which slides on the motor shaft and into the larger tube.
After first using the wrong material for the 3D-printer, which resulted in a messy and unstable part, I successfully printed both parts.

\subsection{Camera Holder}\label{subsec:camera-holder}
The camera holder also consists of two parts: the wooden base and the plastic holder.
I wanted to use wood for the base, as printing a larger rectangle would take a long time and would be a waste of material.
So I used the CNC-Machine to cut out a hole where the 3D-printed part fits into nicely.
The camera holder required two attempts as the first time I didn't make a larger hole for air to flow through to cool the camera.
The second model even had a hole for a small fan.
However, after \("\)carefully\("\) (touching it after letting it run for 10 min) measuring the cooling performance of the camera without a fan I came to the conclusion that the fan wasn't necessary.
