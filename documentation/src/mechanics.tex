\section{Design}\label{sec:design}
The design has to be as simple as possible, because the more parts there are the more things can go wrong.
% maybe insert quote from elon musk lol
The main problem that one encounters when designing such a system is:
As soon as the motor moves the tube, and with it the player, the other motor cannot rotate the player anymore and vice versa.
To solve this problem, I have to design two parts/systems that can solve this problem.
%The first connects two tubes that can rotate freely but if one moves the other one moves with it.
%The second one is the opposite, it connects two tubes that can move freely back and forth but if one rotates the other one rotates with it.
Therefore, I have two main parts, I call them: Move Side and Shoot Side.

\subsection{Move Side}\label{subsec:move-side}
The Move Side is the part that moves the player back and forth.
I use a combination of a gear and a gear rack to move the player.
Additionally, I have a mechanism that connects the two tubes.
A render of the gears and the gear rack is shown in figure~\ref{fig:move_side_gear}.
%\begin{figure}[H]
%    \centering
%    \includegraphics[height=7cm]{../photos/move_side_gear}
%    \caption[moveside1]{Move side gear and gear rack}
%    \label{fig:move_side_gear}
%\end{figure}
\begin{center}
    \begin{figure}[H]
        \centering
        \scalebox{1}{
            \noindent
            \begin{tikzpicture}
                \node [anchor=west] (gear) at (-1,3.8) {\Large Gear};
                \node [anchor=west] (gear-rack) at (-1,6) {\Large Gear rack};
                \node [anchor=west, text width=2cm] (motor) at (-1,2) {\Large Motor \small \textbf{(PD42-3-1141)}};
                \node [anchor=east, text width=2cm] (holder) at (17,5) {\Large Holder \small connects the motor to the wall of the table};
                \node [anchor=east, text width=2cm] (tube) at (17,2) {\Large Tube};
                \begin{scope}[xshift=1.5cm]
                    \node[anchor=south west,inner sep=0] (image) at (0,0) {\includegraphics[width=0.7\textwidth]{../photos/move_side_gear}};
                    \begin{scope}
                        [x={(image.south east)},y={(image.north west)}]
%            \draw[red,ultra thick,rounded corners] (0.48,0.80) rectangle (0.55,0.95);
%            \draw [-latex, ultra thick, red] (note) to[out=0, in=-120] (0.48,0.80);
%            \draw [-stealth, line width=5pt, cyan] (water) -- ++(0.4,0.0);
                        \draw [-stealth, line width=2pt, red] (gear) -- ++(0.65, 0.0);
                        \draw [-stealth, line width=2pt, red] (motor) -- ++(0.60, 0.0);
                        \draw [-stealth, line width=2pt, red] (gear-rack) -- ++(0.35, 0.0);
                        \draw [-stealth, line width=2pt, red] (holder) -- ++(-0.65, 0.0);
                        \draw [-latex, ultra thick, red] (tube) to[out=180, in=-25] ++(-0.33,0.23);
                    \end{scope}
                \end{scope}
            \end{tikzpicture}%
        }
        \caption[moveside1]{Move side gear and gear rack}
        \label{fig:move_side_gear}
    \end{figure}
\end{center}

% Todo: insert a picture of the meachaism that connects the tubes
%\todo{Insert real life image?}

\subsection{Shoot Side}\label{subsec:turn-side}
The Shoot Side is the part that rotates the player.
The turning motor is connected to a tube containing the smaller tube on which the player sits.
The larger tube contains a slit along its length, and the smaller tube has a pin that fits (a screw in this case) into the slit.
A render of the shooting mechanism is shown in figure~\ref{fig:turn_side}.

\noindent
\begin{figure}[H]
    \centering
    \scalebox{0.9}{

        \begin{center}
            \begin{tikzpicture}
%    [
%transform canvas={scale=1}
%]

%        \useasboundingbox (-6.0,-6.3) rectangle (6.0,9.5);
                \node [anchor=west, text width=2cm] (motor) at (-1,4.6) {\Large Motor \small \textbf{(Pololu)}};
                \node [anchor=east, text width=3cm] (larger-tube) at (18.5,9) {\Large Larger tube \small connected to the motor};
                \node [anchor=east, text width=3cm] (main-tube) at (18.5,6.95) {\Large Main tube \small connected to the player};
                \node [anchor=east, text width=3cm] (slit) at (18.5,5.5) {\Large Slit};
                \node [anchor=east, text width=3cm] (pin) at (18.5,3.5) {\Large Pin};
                \node [anchor=east, text width=3cm] (holder) at (18.5,1.5) {\Large Holder \small connects the motor to the wall of the table};
                \begin{scope}[xshift=1.5cm]
                    \node[anchor=south west,inner sep=0] (image) at (0,0) {\includegraphics[width=0.7\textwidth]{../photos/turn_side}};
                    \begin{scope}
                        [x={(image.south east)},y={(image.north west)}]
%                \draw[blue, thick] (-0.2,0) rectangle (1.3,1);
                        \useasboundingbox (-0.2,0) rectangle (1.3,1);
%            \draw[red,ultra thick,rounded corners] (0.48,0.80) rectangle (0.55,0.95);
%            \draw [-latex, ultra thick, red] (note) to[out=0, in=-120] (0.48,0.80);
%            \draw [-stealth, line width=5pt, cyan] (water) -- ++(0.4,0.0);
                        \draw [-stealth, line width=2pt, red] (motor) -- ++(0.25, 0.0);
                        \draw [-stealth, line width=2pt, red] (larger-tube) to[out=180, in=90] ++(-0.6, -0.29);
                        \draw [-stealth, line width=2pt, red] (main-tube) -- ++(-0.21, 0.0);
                        \draw [-stealth, line width=2pt, red] (slit) to[out=180, in=-90] ++(-0.4, 0.08);
                        \draw [-stealth, line width=2pt, red] (pin) to[out=180, in=-90] ++(-0.515, 0.245);
                        \draw [-stealth, line width=2pt, red] (holder) to[out=180, in=-60] ++(-0.715, 0.2);
%            \draw [-latex, ultra thick, red] (tube) to[out=180, in=-25] ++(-0.33,0.25);
                    \end{scope}
                \end{scope}

            \end{tikzpicture}
        \end{center}
    }
    \caption[turnside1]{Shoot side mechanism}
    \label{fig:turn_side}
\end{figure}

%\todo{Insert real life image}
